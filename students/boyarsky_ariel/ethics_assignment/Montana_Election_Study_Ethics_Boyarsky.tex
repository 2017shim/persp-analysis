\documentclass[12pt]{article}

% Any percent sign marks a comment to the end of the line

% Every latex document starts with a documentclass declaration like this
% The option dvips allows for graphics, 12pt is the font size, and article
%   is the style

\usepackage[pdftex]{graphicx}
\usepackage{url}
\usepackage[margin=1in]{geometry} 
\usepackage{amsmath,amsthm,amssymb,amsfonts,enumerate,lipsum}
\usepackage[utf8]{inputenc}
\usepackage[superscript,biblabel]{cite}

\newenvironment{theorem}[2][Theorem]{\begin{trivlist}
\item[\hskip \labelsep {\bfseries #1}\hskip \labelsep {\bfseries #2.}]}{\end{trivlist}}
\newenvironment{lemma}[2][Lemma]{\begin{trivlist},
\item[\hskip \labelsep {\bfseries #1}\hskip \labelsep {\bfseries #2.}]}{\end{trivlist}}
\newenvironment{exercise}[2][Exercise]{\begin{trivlist}
\item[\hskip \labelsep {\bfseries #1}\hskip \labelsep {\bfseries #2.}]}{\end{trivlist}}
\newenvironment{reflection}[2][Reflection]{\begin{trivlist}
\item[\hskip \labelsep {\bfseries #1}\hskip \labelsep {\bfseries #2.}]}{\end{trivlist}}
\newenvironment{proposition}[2][Proposition]{\begin{trivlist}
\item[\hskip \labelsep {\bfseries #1}\hskip \labelsep {\bfseries #2.}]}{\end{trivlist}}
\newenvironment{corollary}[2][Corollary]{\begin{trivlist}
\item[\hskip \labelsep {\bfseries #1}\hskip \labelsep {\bfseries #2.}]}{\end{trivlist}}
% These are additional packages for "pdflatex", graphics, and to include
% hyperlinks inside a document.

\setlength{\oddsidemargin}{0.25in}
\setlength{\textwidth}{6.5in}
\setlength{\topmargin}{0in}
\setlength{\textheight}{8.5in}

% These force using more of the margins that is the default style

\begin{document}

% Everything after this becomes content
% Replace the text between curly brackets with your own

\title{The Ethics of the 2014 Montana Election Study}
\author{Ari Boyarsky \\ aboyarsky@uchicago.edu}
\date{November 20, 2017}

% You can leave out "date" and it will be added automatically for today
% You can change the "\today" date to any text you like


\maketitle

% This command causes the title to be created in the document

\section{Introduction}

The October 2014 Montana election study conducted by Kyle Dropp, Jonathon Rodden, and Adam Bonica meant to study voter turnout, exemplifies ethically divisive experiment that may have caused public and personal harm to individuals. The study employed letters to individuals, selected by political ideology, that provided information on the political leanings of a judges up for election to the Montana supreme court. It should be noted that this election is non-partisan. The study, which assessed the impact of partisanship on voter turnout generated outrage and condemnation amongst the broader public. We utilize the ethical framework established in Matthew Salganik’s Bit by Bit to assess the ethics of this study. Overall, we show that this study represents a violation of Salagnik’s standards.  

\section{Four Principles of Ethical Research}

Salganik identifies four principles of ethical research that he believes are highly pertinent in todays digital age. The ideas are based on the Belmont and Menlo report. The characteristics they identify are respect for persons, beneficence, justice, and respect for law and public interest. We will use these four criteria in evaluating the ethical standing of the Montana experiment. 

\subsection{Respect for Persons}

The first principle that Salganik notes is respect for persons. This principle is taken to mean that researches must receive informed consent from participants. This is based on the idea that individuals are autonomous, that humans are responsible for themselves. Furthermore, individuals who are not entirely responsible for themselves (i.e. children under the age of 18) should be treated with even greater care. Hence, participant consent is key to ethical research.

Applying this to the Montana election experiment it is clear that Rodden, Bonica, and Dropp did not follow the principle of respect for persons. First and foremost, individuals who were exposed to the treatment were not informed and their consent was not ascertained. Furthermore, the treatment was applied without any clear consideration for individuals with lessened autonomy. Rodden et al. could not know if an individual receiving the flier was mentally ill, or disadvantages in any other way. Furthermore, it is very possible that the treatment diminished individual autonomy in casting a vote. That is, by displaying a conservative liberal scale with potentially polarizing individuals such as President Barack Obama and Governor Mitt Romney could have greatly influenced how individuals voted. This is especially concerning given that the election was non-partisan.  However, the key concern is that by displaying electoral candidates on a ideological scale (that may or may not be valid) that contained possibly polarizing national politicians may have unfairly influenced voters.

Of course, one can also argue that this was not a breach of respect for persons. This argument might assume that individuals are responsible and able to make their own decisions based on information. That is, that this new information is no different than any other non-partisan ad a voter may encounter. However, it should be noted that Rodden et al. wanted to instigate a reaction. Their study was predicated on a possible reaction voter may have. Additionally, they did this without thinking about individuals who may receive the mailers who are not entirely autonomous. Finally, non-partisan usually need approval of election authorities. Rodden et al. did not seek this provision according to Jonathan Motl’s report. Furthermore, the use of the state seal is even more misleading, possibly making some individuals feel these views were endorsed by the state. Therefore, it is not easy to argue that this study did follow the principle of respect for persons.

\subsection{Beneficence}

Next, Salganik discusses the principle of beneficence which he bases on the idea presented in the Belmont Report. Specifically, this is intended to mean that a study does not harm individuals where possible, or at least maximizes benefits and reduces harms. It is not as clear whether the Rodden et al. study followed this principle, as it is not clear at first glance what the exact benefits and harms of the study were. This is furthered as Salganik argues that harm could be excused if the study presents a convincing benefit. Based on this we can clearly say that the Rodden et al. study did not follow this principle.

Specifically, the study subject all its participants to harm which is the manipulation of their vote with almost no benefit. The point of the study was to measure partisanship and voter turnout; however the same study could have been accomplished via a natural experiment or at least a smaller scale one (we will discuss these options in more detail in the final section). Indeed, the manipulation of voters is like disenfranchisement which is certainly a political harm to an individual. Furthermore, it is not clear that Rodden et al. engaged in the cost benefit analysis that Salganik suggests. Hence, it does not appear that the study follows this principle.

While it may be argued that Rodden et al. did engage in a costs benefit analysis which favored their study, it is difficult to identify exactly what benefit the study presents that outweighs the cost. Once, we consider manipulation of voters and an election to be a harm, the benefit of the study must now pass a high bar. Especially since if the mailers did influence the election, there may be a broad public harm as the preferred candidate may not have won the election. Hence, it is difficult to argue that the authors engaged in a fair cost/benefit analysis. Additionally, it is not obvious that the researchers took any steps to minimize risk to individuals.

\subsection{Justice}

Salganik’s principle of justice is also based on the Belmont Report which draws from several works. Initially, this principle was focused on the protection of vulnerable individuals. It has also come to mean the inclusion of marginalized groups. However, it has also come to mean proper compensation for participation in studies. Based on which of these definitions we choose to use to evaluate this study we may find different results. 

First, if we use the initial definition then it is clear the researchers did not heed this principle. Specifically, because they caused harm to multiple individuals without considering if they are a vulnerable population and the impact that this may have on their broader community. However, if look at the second definition, their loose specification of individuals may mean they did include a fair distribution of vulnerable groups so that their actions may also inform the study. Finally, none of the participants were compensated in the study, hence the researchers did not follow this principle. 

While one could argue that the researchers followed the second definition of justice – inclusion of marginalized groups. The research design did not specifically intend to do this. Indeed, they focused on political ideology not on minority status. Hence, this was not an intentional design, so it is difficult to credit the researchers for considering this principle. Thus, overall the researchers did not follow the principle of justice as outlined by Salganik. 

\subsection{Respect for Laws and Public Interest}

Salganik’s final principle respect for law and public interest is based on the Menlo report. This principle is focused on the compliance and transparency of the study and researchers. Compliance simply means that the researchers complied with relevant laws. While transparency entails that researchers reveal their goals and methods and take responsibility for their actions. Clearly, Rodden et al. did not follow these standards.

Evaluating this criterion is less ambiguous. Specifically, the Motl ruling (2015) found that the researchers did not comply with registration and disclosure laws of Montana that are usually in place for non-partisan political advertising. While it could be claimed that there was no political purpose to the mailers, other entities would be held to this standard and so Rodden et al. should as well. Also, intention is not necessarily the crux of these laws, since the mailers could have had a political effect they should have been registered. Also, the use of the state seal was also further misleading and may have also violated laws. It was also an obvious example in which the researchers overlooked legal compliance. Of course, Rodden et al. also did not take responsibility for the study, as after the Motl report was released Stanford disagreed with the ruling and ignored the study except for an apology letter sent to participants. This however, does not absolve Rodden et al. of all responsibility of the impact of the study.  Furthermore, the fact that this study may have influenced an election further shows how little the researchers considered the consequences of this study. Specifically, if the election results were changed due to this study it is possible it was not done so in the public interest. Hence, this study also harmed the public interest. Motl (2015) notes that the supreme court is essential to the daily lives of Montanans and thus any influence this study exerts on the electoral outcome also translates to a far broader impact on Montanans. 

\section{Alternative Design: Randomized Mailers}

Had this study employed a randomized treatment assignment to individuals the election could have been similarly impacted. Specifically, the mailers pointed out the partisanship of the candidates in a non-partisan election. Hence, by placing focus on this aspect of the candidates the study may have impacted why voters voted the way they did. This is especially true since national politicians were also displayed on the mailer. Thus, insinuating a connection between the candidates and the politicians. Now, voters may choose who to cast their vote for based on their views of that politician and not the candidate. Hence, the effect of the study on the election would be based on random individuals who received the mailers. Regardless of direction, the study would have some effect as Motl (2015) points out the election could easily have been influenced by the mailers. Thus, the study could easily have impacted the composition of the Montana supreme court.

\section{Actual Research Design}

The actual design of the study, which sent out 64,265 mailers to liberals and 39,515 mailers to conservatives does not change much about the impact or ethicality of the study. Indeed, this has just as much power in changing election results as a randomized treatment. In fact, the high liberal to conservative ratio could influence the election more, since democrats will be impacted in one way or another more than republicans. In terms of ethics, it does not change how we view the four ethical principles set forth by Salganik. Again, if anything it may be less just and have less respect for individuals since democrats seem to be targeted for no apparent reason other than possible low turnout. 

\section{Research Design Choices: Low Contestation}

While their concern for a distorted outcome does show that some care was taken in choosing the election and thus the public interest was taken into account, it does not change our overall opinion of the ethical aspects of the study as a whole. First, the primary is not necessarily indicative of the competitiveness of an election. Furthermore, as we will discuss in the last section there are other research designs that could have been employed. Furthermore, even if they changed only a single voter’s decision they caused unnecessary harm by diminishing that individuals autonomy. Hence, respect for individuals, beneficence, and justice do not change. Furthermore, even with these considerations they still did not comply with state laws and their use of the seal was misleading at best and egregiously illegal at worst. Hence, our view of this studies respect for laws and public interests does not change.

\subsection{Results of the Election}

The results of the election again do not change our view. Specifically, as we stated above even if one person was impacted this study caused harm by reducing individual autonomy. This is true no matter how close the race was, or if the actual election results were changed. Indeed, over 90,000 individuals were impacted and the basic facts that the study did not comply with state laws does not change (Motl, 2015). Thus, the election results do not change our condemnation of this study.

\section{Adam Bonica and CrowdPAC}

Bonica is also involved with a for-profit company called CrowdPAC. The site helps organize communities and collective action, it also calculates ideological scores for politician. Bonica used some of this data to calculate the judge candidates scores. While this does not immediately change anything about our views of the study, it could indicate a far greater misstep. Specifically, if Bonica or CrowdPAC profited in some way from the study. If so, this would be an even greater violation of respect for laws and pubic interest as well as respect for individuals. This is because it would most likely violate some state or federal law if Bonica profited from the study, and it would show even more disregard for individuals as he was willing to risk causing individual harm not for scientific advancement and benefit but for personal gain and profit. Hence, while this does not change our previous views, it may possibly warren further investigation.   

\section{Possible Improvements and Alternative Designs}

There are a plethora of ways that Rodden et al. could have studied electoral turn out and partisanship without violating the ethical principles that Salganik puts forth. We will look at two alternatives that would fulfill this purpose. First, we look at an experimental procedure that would avoid these ethical qualms. Next, we discuss a natural experiment that would avoid these harms. 

The first option we look at is an experimental design that could also be used to test how partisanship impacts voter turnout. This could be done over a digital medium such Amazon’s MTurk or an in person. We will discuss the MTurk option because it offers advantages such as being more efficient and cheaper. Specifically, we present a survey taker a page that asks for political ideology, party affiliation, and demographic data. We then present them with a choice of two candidates for a non-partisan position such as a state supreme court judgeship. For half the participants this is all they will see. They will then be asked to vote for one of the candidates or abstain. For the other half of participants we show them the Rodden et al. scale (i.e. where each candidate falls with respect to prominent politicians). We offer them the same choices. While, this experiment is not perfect, there are low stakes, individuals may not be paying too much attention, and individuals make just vote randomly. However, we suspect that over a large enough sample size we may see a difference in the number abstentions between the control and treatment group. Furthermore, given the limitations in this design any difference between the two groups that passes a basic difference of means test at a significance level of 95\% ($\alpha = 0.05$) would be very powerful. Indeed, this basic standard of significance for a difference in population is probably enough to infer a strong correlation between partisanship and voter turnout.

The next design we consider employs an observational study of a natural experiment. Indeed, while the Montana judgeship is a non-partisan election, this is not true of all judgeship elections. For instance, New York employs a partisan supreme court election that consists of party convention delegates choosing the candidates for the general election. This is also true of other states such as Michigan, Ohio, Pennsylvania, and others (Goldberg, 2002). Rodden et al. would simply have to collect historical time-series data for this election and another state which employs a non-partisan election (like Montana), and compare the turn outs. Of course, there are many issues with comparing these heterogeneous groups. However, much of this could be overcome with proper statistical adjustment. Perhaps correcting for turnout in both states based on turnout for presidential and mid-term elections, as well as correcting for population differences. Other heterogeneities such as electoral legal differences, and other factors will still discount the results but will not disqualify the study entirely. That is, useful inference could be made. 

Overall, these alternative research designs show that it very possible to measure how partisan information can impact a non-partisan election, without influencing an election, manipulating individuals, and skirting state electoral laws. Indeed, while these studies do have some flaws as we mentioned above, they avoid almost all the ethical missteps of the original study by studying non-reactive observational data, and soliciting the consent of individuals and offering compensation. It is important to know that the initial study also had methodological problems. For instance, by distributing these fliers the researchers could not be sure they would influence decisions. Perhaps, voters simply threw them out the mailers. Perhaps, they had heterogeneous effects (i.e. some turned out other did not due to the mailers). Thus, we show it is possible to study the phenomena without the same ethical dilemmas. 


\newpage


\bibliographystyle{unsrt}
\bibliography{ethics_bib}
\nocite{*}



\end{document}